\section{Vektoren}

\textit{$\bullet$ Skalarprodukt} \\
\textit{$\cdot$ Matrixprodukt}

\subsection{Addition}

$\vec{a} + \vec{b} = \begin{bmatrix}
    a_1 \\
    a_2 \\
    a_n \\
\end{bmatrix} + \begin{bmatrix}
    b_1 \\
    b_2 \\
    b_n \\
\end{bmatrix} = \begin{bmatrix}
    a_1 + b_1 \\
    a_2 + b_2 \\
    a_n + b_n \\
\end{bmatrix}$

\subsection{Multiplikation mit Skalar}

$\lambda \vec{a} = \lambda\begin{bmatrix}
    a_1 \\
    a_2 \\
    a_n \\
\end{bmatrix} = \begin{bmatrix}
    \lambda a_1 \\
    \lambda a_2 \\
    \lambda a_n \\
\end{bmatrix}$ \\

\textit{$\lambda \in$ Skalar}

\subsection{Skalarprodukt}

$\vec{a} \bullet \vec{b} = |\vec{a}| \cdot |\vec{b}| \cdot \cos \pi$

\subsection{Orthogonal}

$\vec{e}_x \bullet \vec{e}_y = 0$

\textit{
    Senkrecht zueinander, wenn Skalarprodukt zweier Einheitsvektoren 0 ergibt.
}

\subsection{Länge des Vektors}
$ \norm{v} = \sqrt{v \cdot v} $

\subsection{Einheitsvektor}

$ e_v = \frac{1}{\norm{v}} \bullet v $ \\
$ ( i = e_1, j = e_2, k = e_3 ) $ \\

\textit{$\vec{e}_x = [1,0,0]^T$}\\
\textit{$\vec{e}_y = [0,1,0]^T$}\\
\textit{$\vec{e}_z = [0,0,1]^T$}

\subsection{Euklidische Distanz}

$\bar{AB} = \sqrt{(b_1-a_1)^2 + (b_2 - a_2)^2 + \dots + (b_n - a_n)^2}$

\subsection{Achsenabschnitt}

\textit{Gegeben sind 3 Punkte $p_x = x$,$p_y = y$,$p_z = z$ ergibt Ebenegleichung:} \\

$\frac{x}{p_x} + \frac{y}{p_y} + \frac{z}{1} = 1$,
HNF = $\frac{}{}$

\subsection{Hessische Normalform}

TODO

\subsection{Begriffe}

\begin{tabular}{r|l}
    \textbf{Ortsvektor}         & Vom Ursprung zum Punkt \\
    \textbf{Richtungsvektor}    & Eine Richtung im Raum \\
    \textbf{Einheitsvektor}     & Eine Einheit in eine beliebige Richtung \\
    \textbf{Linearkombination}  & Ein Vektor, der ein vielfaches \\
                                & eines Einheitvektors ist \\
    \textbf{Skalar}             & Ist ein reelle oder komplexe Zahl \\
    \textbf{Rechtssystem}       & Koordinatensystem aufgebaut wie die \\
                                & rechte Hand wobei; der Zeigfinger \\
                                & X-Achse ($\vec{e}_x$), Mittelfinger Y-Achse ($\vec{e}_y$) \\
                                & und Daumen Z-Achse ($\vec{e}_z$)
\end{tabular}