\section{Transformation}

\subsection{Transformation des Koordinatenystems}

TODO

\subsection{homogene Koordinaten}

\textit{jeder Punkt P(x,y,z) des Raumes $\mathbb{R^3}$ besitzt eine 4-komponenten Vektor $\vec{r}$}
\\
$\vec{r} = \begin{bmatrix}
    x_1 \\
    x_2 \\
    x_3 \\
    x_4
\end{bmatrix}$, 
$x = \frac{x_1}{x_4}$, 
$y = \frac{x_2}{x_4}$, 
$z = \frac{x_3}{x_4}$ \\

$(x,y,z) = (\frac{x_1}{x_4}, \frac{x_2}{x_4}, \frac{x_3}{x_4})$

\subsection{Ebene im Raum}

\textit{Ebene $\epsilon$ im Raum $\mathbb{R}^3$}
\\
$\epsilon : ax + by + cz + d = 0$
\textit{Hessische Normalform}
\\ \\
$\vec{w} = \begin{bmatrix}
    a \\
    b \\
    c \\
    d
\end{bmatrix}$, Punkt:
$\vec{r} = \begin{bmatrix}
    x \\
    y \\
    z \\
    1
\end{bmatrix}$ \\

\textit{Ebenengleichung:}

$\vec{w} \bullet \vec{r} = w^T \cdot r = ax + by + cz + d = 0$ \\

\subsection{Prokektive Transformation}

\textit{Die homogene Matrix $\mathbf{H}$ ist nur bis auf einen konstanten \
 Faktor bestimmt, heisst, alle Vielfachen von $\mathbf{H}$ sind auch gültig} \\

\text{$\eta : \mathbb{P}^3 \mapsto \mathbb{P}^3 $ stellt eine \textbf{projektiven Transformation} dar} \\

$\eta(r) = \mathbf{H} \cdot r = \begin{bmatrix}
    h_{11} & h_{12} & h_{13} & h_{14} \\
    h_{21} & h_{22} & h_{23} & h_{24} \\
    h_{31} & h_{32} & h_{33} & h_{34} \\
    h_{41} & h_{42} & h_{43} & h_{44}
\end{bmatrix} \cdot \begin{bmatrix}
    x_1 \\
    x_2 \\
    x_3 \\
    x_4 
\end{bmatrix}$

\textbf{Euklidisch} (starre Bewegung) \\
$D = \begin{bmatrix}
    \mathbf{R} & t \\
    0^T & 1
\end{bmatrix}$ \\
\textit{Abstand zwischen zwei Punkten, alle Winkel} \\
\textit{($R^{-1} = R^T$)} \\

\textbf{Ähnlichkeit} \\
$S = \begin{bmatrix}
    k \cdot \mathbf{M} & t \\
    0^T & 1
\end{bmatrix}$ \\
\textit{Winkel zwischen zwei Punkten, alle Winkel} \\

\textbf{Affin} \\
$A = \begin{bmatrix}
    \mathbf{C} & t \\
    0^T & 1
\end{bmatrix}$ \\
\textit{Parallelität, Verhältnis zwischen Volumeninhalt} \\

\textbf{Allgemein} \\
$\mathbf{H} = \begin{bmatrix}
    h_{11} & h_{12} & h_{13} & h_{14} \\
    h_{21} & h_{22} & h_{23} & h_{24} \\
    h_{31} & h_{32} & h_{33} & h_{34} \\
    h_{41} & h_{42} & h_{43} & h_{44}
\end{bmatrix}$ \\
\textit{Geraden bleiben Geraden} \\

\subsection{Euklidische Transformationen}

TODO Translation, Spiegelung an einer Ebene, Rotation, Zusammensetzen von

\subsection{Rotation um beliebige Achse}

1) Rotation um $\phi$ um z-Achse (Matrix D) \\
2) Rotation um den Winkel $\theta \in [0, \pi]$ (um frühere X-Achse) (Matrix C)  \\
3) Eigentlich Rotation um den gegeben Winkel $\psi$ (Matrix B) \\

\textit{$c_\alpha = \cos \alpha$, $s_\alpha = \cos \alpha$, $\alpha \in {\phi, \theta, \psi}$} \\

\newcolumntype{C}{>{\centering\arraybackslash} m{6cm} }
\begin{tabular}{m{3cm}CC}
    \includegraphics[scale=0.3]{rotation_matrix_D} &
    $\mathbf{D} = \begin{bmatrix}
        c_\phi & s_\phi & 0 \\
        -s_\phi & c_\phi & 0 \\
        0 & 0 & 1 
    \end{bmatrix}$ \\
\end{tabular} \\
\begin{tabular}{m{3cm}CC}
    \includegraphics[scale=0.3]{rotation_matrix_C} &
    $\mathbf{D} = \begin{bmatrix}
        1 & 0 & 0 \\
        0 & c_\theta & s_\theta \\
        0 & -s_\theta & c_\theta
    \end{bmatrix}$ \\
\end{tabular} \\
\begin{tabular}{m{3cm}CC}
    \includegraphics[scale=0.3]{rotation_matrix_B} &
    $\mathbf{D} = \begin{bmatrix}
        c_\psi & s_\psi & 0 \\
        -s_\psi & c_\psi & 0 \\
        0 & 0 & 1
    \end{bmatrix}$ \\
\end{tabular}

Danach wieder zurück rotieren um $\phi$ und $\theta$

\subsection{Rotation um eine Achse durch den Ursprung}

TODO insert T / $R_{y,x,z}$ \\

Todo rotation around any axis \\

Todo altertative, rotation around origin \\

\subsection{Parallele Projektion}

\textit{Projektion auf Ebene $\epsilon: ax + by + cz + d = 0$} \\
\textit{Die ebene ist definiert durch Normalvektor $\vec{n} = \begin{bmatrix}
    a \\
    b \\
    c
\end{bmatrix}$} \\
\textit{Normalenvektor erhalten: $|\vec{n}| = \sqrt{a^2 + b^2 + c^2} = 1$} \\

\textit{Projektionsrichtung definiert durch $\vec{v} = (v_x, v_y, v_z)$} \\
\textit{Normalisieren von Projektionsrichtung: $|\vec{v}|$} \\

\textit{Ist $|\vec{n}|$ (Ebenen Normalenvektor) und $|\vec{v}|$ (Projektionsrichtung) gegeben} \\

$\vec{x} = \vec{x}_0 + t\vec{v}$, komponentenweise $\begin{bmatrix}
    x = x_0 + tv_x \\
    y = y_0 + tv_y \\
    y = y_0 + tv_y
\end{bmatrix}$ \\
\textit{Wobei $x_0$ Punkt wo auf $x$ auf Ebene Projeziert wird} \\

\textit{$\psi$ entspricht Winkel zwischen $\vec{n}$ und $\vec{v}$} \\

$cos(\psi) = \vec{v} \bullet \vec{n}$

TODO - gleichung t t* \\

\subsection{Parallele Projektionsmatrix}

$\begin{bmatrix}
    x^* \\
    y^* \\
    z^*
\end{bmatrix} = \mathbf{H} \begin{bmatrix}
    x_0 \\
    y_0 \\
    z_0
\end{bmatrix} = $ \\
$\frac{1}{c_\psi} \begin{bmatrix}
    (c_\psi - av_x) & -bv_x & -cv_x & -dv_x \\
    -av_y & (c_\psi - bv_y) & -cv_y & -dv_y \\
    -av_z & -bv_z & (c_\psi - cv_z) & -dv_z \\
    0 & 0 & 0 & c_\psi
\end{bmatrix} \begin{bmatrix}
    x_0 \\
    y_0 \\
    z_0 \\
    1
\end{bmatrix}$ \\
\textit{$cos(\psi) = c_\psi$}

\subsection{Perspektivische Projektion}

\textit{Fall wenn Zentrum $O$ im Nullpunkt} \\

$\epsilon: ax + by + cz + d = 0$, Ebene \\

\textit{Beliebigen Punkt $A_0(x_0,y_0,z_0)$ mit Projektionspunkt $A^*(x^*,y^*,z*)$ in Ebene $\epsilon$} \\

$\begin{bmatrix}
    x^* \\
    y^* \\
    z^* \\
\end{bmatrix} = \begin{bmatrix}
    \lambda x_0 \\
    \lambda y_0 \\
    \lambda z_0 
\end{bmatrix}$ \\

$\lambda = - \frac{d}{ax_0 + by_0 + cz_0}$ \\

$(ax_0 + by_0 + cz_0) \cdot \begin{bmatrix}
    x^* \\
    y^* \\
    z^* \\
    1
\end{bmatrix} = \begin{bmatrix}
    -dx_0 \\
    -dy_0 \\
    -dz_0 \\
    ax_0 + by_0 + cz_0
\end{bmatrix} = \begin{bmatrix}
    -d & 0 & 0 & 0 \\
    0 & -d & 0 & 0 \\
    0 & 0 & -d & 0 \\
    a & b & c & 0
\end{bmatrix} \begin{bmatrix}
    x_0 \\
    y_0 \\
    z_0 \\
    1
\end{bmatrix}$ \\

\subsection{Perspektivische Projektionmatrix}

$\mathbf{H} = \begin{bmatrix}
    -d & 0 & 0 & 0 \\
    0 & -d & 0 & 0 \\
    0 & 0 & -d & 0 \\
    a & b & c & 0
\end{bmatrix}$ \\

\subsection{Sichtvolumen Clipping}

\textit{Das kanonische Sichtvolmen ist ein Würfel mit $P(\pm 1, \pm 1, \pm 1)$} \\
\textit{Defür sind vorne und hinten, sowie zwei Punkte bestimmend Grösse gegeben} \\

\includegraphics[scale=0.4]{clipping} \\

\textit{$P$ links unten, $Q$ rechts oben} \\
\textit{$z$ vorne $z=-a$, $z$ hinten $z=-b$} \\

$\mathbf{T} = \begin{bmatrix}
    \frac{2a}{x_Q - x_P} & 0 & \frac{x_Q + x_P}{x_Q - x_P} & 0 \\
    0 & \frac{2a}{y_Q - y_P} & \frac{y_Q + y_P}{y_Q - y_P} & 0 \\
    0 & 0 & -\frac{b+a}{b-a} & -2\frac{ba}{b-a} \\
    0 & 0 & -1 & 0 \\
\end{bmatrix}$
